\documentclass[11pt, a4paper, titlepage]{jarticle}

% VSCode で Tex を実行中
% 環境構築は
% https://qiita.com/Gandats/items/d7718f12d71e688f3573

% レポート書くときに
% https://qiita.com/birdwatcher/items/5ec42b35d84d3ee2ffbb


% ===== パッケージ =====

% 数式モード
\usepackage{amsmath}
\usepackage{amsfonts}
\usepackage{mathtools}
\usepackage{enumerate}
\usepackage{bm}

% 画像
\usepackage{graphicx}
\usepackage[dvipdfmx]{color}

% ハイパーリンク
\usepackage[dvipdfmx]{color, hyperref}
\usepackage{pxjahyper}
\usepackage{cite}
\usepackage{xcolor}
\hypersetup{colorlinks=false}

% コメントアウト
\usepackage{comment}

\title{タイトル}
\author{名前}

\西暦
\date{\today}

\begin{document}

\maketitle
\tableofcontents
\clearpage

\section{セクション1}

\begin{comment}

% 画像2つバージョン
\begin{figure}[htbp]
    \begin{minipage}{0.5\hsize}
     \begin{center}
      \includegraphics[width=50mm]{graph01.jpeg}
     \end{center}
     \caption{caption1}
     \label{fig:one}
    \end{minipage}
    \begin{minipage}{0.5\hsize}
     \begin{center}
      \includegraphics[width=50mm]{graph02.jpeg}
     \end{center}
     \caption{caption2}
     \label{fig:two}
    \end{minipage}
\end{figure}

\end{comment}

\subsection{サブセクション}
hoge

\section{セクション2}
hogehoge

\end{document}